\section{Verplank diagram}
\label{Verplank_diagram}
over the course of designing the fingerdrums a Verplan digram was created to help specify and explain the idea. 
\subsection{Idea}
Playing the drums on any surface using only your fingers as drumsticks.

\subsection{Metaphore}
Having a drum set in the palm of your hand that you can play at any time, in any place.

\subsection{Model}
Play the drums without being anywhere near an actual drum set.

\subsection{Display}
Small OLED display on the wrist which shows the type of drum set currently being used.

\subsection{Error}
Any touchscreens, for instance, your phone will not be usable when waring the glove. Drumming on any object will still produce a sound, which could be disruptive, depending on the object.

\subsection{Scenario}
You can play th drums on the bus on the way home from work without having to bring a actual drum set.

\subsection{Tasks}







\section*{Introduction}
\label{Introduktion}
This paper has been written for the course \textit{Prototyping and Fabrication Techniques} at Aalborg University. It describes the design process and the making of a conceptual prototype which will allow the user to have a simple drum kit at their fingertips. The prototype ended up using five Force Sensitive Resistors (FSR) attached to a glove, and an Arduino Uno with a custom shield made to fit the FSRs properly. The Arduino communicated the signals form the FSRs to a PC via the Firmata Protocol where Processing was used to analyse the signals and play the drum sounds. The paper contains a further work section which will describe the current problems and propose the next possible iteration in order to address some of these problems.