\chapter{Further Work}
\label{further_work}
Nogle ideer til further work:
videre arbejde - trådløs - mixer (midi/equalizer/garageband/sampling) - batteri (med beregning) - ergonomi -  processor (lav pcb med processer) - lydkort - hukommelse - interface (hvis ikke knapperne selv så en skærm) - placering af sensorer (håndfladen) - flere sensorer (kompas accell gps) - hovedetelefoner/højtaler m. forstærker. - To handsker + fødder -    

The finger drums prototype is in its current state, dependent of long wires connecting the FSR's to the Arduino, and the arduino to the computer. One of the most obvious improvements for the prototype, would be to make it wireless. This however requires a number of additions. Firstly the computer running the system should be located on the users body. This could be either a smartphone or maybe in form of a bracelet located on the wrist. Regardless some form of wireless communication is needed to avoid the long wires running along the body of the user. The computer should be able to contain the music files and therefore should have a form of storage unit, like an micro SD card for saving the files. For running a computer without an external power source, a battery is required which could be a form of LiPo battery that is rechargeable so that the user doesn't have to change battery. For playing the sounds both a soundcard and a speaker or a set of headphones is required to play the music, either out loud or in private. This would make a bluetooth module an ideal addition to facilitate communication with a phone or wireless headset.
If the prototype is connected to a screened device like a smartphone, interactions like changing drum kits and adjusting volume and sensitivity, as well as the storage of soundfiles etc. could be moved there to save on power consumption and manufacturing cost.
