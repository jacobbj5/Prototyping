\section{Electronics}
\label{electronics}
The finger drums prototype will consist of 5 force sensitive resistors connected to the analog pins of an Arduino Uno. The FSRs used in this prototype will be an FSR-152. In order to get the desired output signal from the FSR it is connected with a normal resistor to create a voltage divider as seen in \autoref{fig:simple_voltage_divider}. 
\begin{figure}
\centering
\includegraphics[scale=1.5]{Figure/simple_voltage_divider.png}
\caption{Simple voltage divider with the FSR represented as a variable resistor. }
\label{fig:simple_voltage_divider}
\end{figure}

In order to find a fitting value for the resistor R the formula for a voltage divider is used, \autoref{fig:R_calculation}.
\begin{figure}
\centering
\includegraphics[scale=0.15]{Figure/R_calculation.png}
\caption{Calculation of a voltage divider.}
\label{fig:R_calculation}
\end{figure}

The input voltage from the Arduino is 5V, the FSR varies in the range 1 M Ohm to 1 K Ohm, and the ideal output voltage is around 0V and 5V depending on the value of the FSR. From \autoref{fig:R_calculation} a resistor of 100 K Ohm is found to be fitting resulting in an output voltage that will range form 4.95V to 0.455V, assuming all components are ideal. The final circuit diagram, all components included can be seen in \autoref{fig:Schematic2}. 
\begin{figure}
\centering
\includegraphics[scale=0.5]{Figure/Schematic2.png}
\caption{Schematic for the final circuit with the blue box indicating which components are actually on the shield.}
\label{fig:Schematic2}
\end{figure}